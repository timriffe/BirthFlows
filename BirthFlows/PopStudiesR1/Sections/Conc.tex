We offer a visualization of the time series of Swedish birth counts, structured by two time measures: year of occurrence (period) and mothers' cohort. We index the same series twice, once to period (top) and once to cohort (bottom), which leads to a reflected set of axes, aligned to a single calendar. Each series consists in a set of sequentially stacked distributions. The period-indexed time series is a set of stacked distributions binned by mother cohorts. The cohort-indexed series is of stacked distributions binned by occurrence years. Each stacked distribution could be summarized in many ways, and we opted to color the distributions by the birth-weighted standard deviation of the maternal age at birth. The color pattern shows a transition to more compact birth distributions after the baby boom. Rather than \emph{detrending} the series, we inject the series with a trend embedded in the calendar $x$-axis, which helps reveal the long-term pattern in crude generation replacement. A maternal lineage centered on Alva Myrdal in rendered atop the series to orient the viewer's interpretation of the visualization, as is a timeline of selected demographic shocks. In the discussion we highlight a few stylized themes and questions, omitting others for the sake of brevity. 

Demographers are used to exploratory or explanatory graphs focused on a single diagnostic or pattern. Exploratory graphs are usually of a familiar form, and are quick to interpret, while explanatory graphs are further distilled to deliver a clear message. Our visualization is neither of these, but rather an excuse to give pause and reflect on the fundamentals of population renewal. The investment required to understand this visualization ought to draw the reader deeper into the joys and frustrations of demographic thought and practice. 
