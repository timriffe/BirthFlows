We would like for our visualization to be an exposition on human renewal in general, but this work will be most useful if based on a long data series of good quality, ergo a single population with its own history and peculiarities. As \citet{perozzo1880della}, we base our exposition on Sweden because the data are available and of adequate dimension, but there are two ways in which this fact is not to be taken for granted: i) The data comes from multiple sources and formats, it varies in quality, and it must be brought to a common Lexis resolution  (i.e., a uniform grid of period and cohort bins) to be useful, and ii) although the visualization shows a history of birth in Sweden in particular, it may serve as a parable for multiple demographic models of a general nature. 

The visualization is based on birth count data from Sweden in single-year bins by year of occurrence (period) and mother cohort. Birth counts for years 1736 to 1750 are reconstructed from a variety of sources \multicitep{HFC, ({HFC}); HMD,({HMD}); sweden1969historisk} using indirect methods (see App.~\ref{app:retroject}). Data for years 1751-1774 are derived via adjustment from HFC estimates and HMD exposures and birth totals (see App.~\ref{sec:hfc}). Data for the period 1775 to 1890 come from \citet{sgf1907}, which we have graduated from mostly five-year age groups into single ages (see App.~\ref{sec:sgf}). The merged birth count series for years 1736 to 1890 is subject to a global adjustment to retain information on cohort size in period birth distributions (see App.~\ref{sec:histadj}). Data for the years 1891 to 2016 is taken directly from the \citet{HFD} (HFD) without further adjustment, although the HFD itself did split single age birth counts into Lexis triangles for years 1891 to 1969 \citep{persson2010human}. To complete our picture, we project the fertility of cohorts whose fertility careers were still incomplete as of 2016 (1962-2016) through age 55 (see App.~\ref{sec:proj}). These steps are fully reproducible, and further details can be found in an open data and code repository.
