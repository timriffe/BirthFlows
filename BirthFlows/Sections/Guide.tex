This section serves as a guide to interpreting our primary visualization (fold-out Fig.~\ref{fig:foldout}). We first describe the primary visual attributes so that the reader may seek meaning in patterns, and we then narrate some of the more salient features of the graph.

\paragraph{Meandering baseline} We start with the stacked polygon representation in Fig.~\ref{fig:joinbins}, and subject it to a single $y$ transformation before re-plotting in our primary figure. This transformation is a $y$ shift proportional to a smoothed time series of the crude cohort replacement ratio (see App.~\ref{sec:baseline} for details), which in effect creates a meandering baseline rather than a flat $x$-axis. We call this crude replacement because the ratio takes no account of mortality or migration. A horizontal white line crosses the middle region of the plot showing where the baseline would have to be to indicate exact crude replacement. Regions where the baseline is above the horizontal cut point, such as the first half of the \nth{19} century, are cohorts that produced a crude surplus of offspring, and vice versa. This transformation distorts the series somewhat, but we think in such a way as to make long term macro patterns more visible without losing the ability to compare proximate points on the calendar.

\paragraph{Profile} The height of each series vis-\`a-vis the baseline represents the total births in each year on the top series and the total cohort offspring on the bottom series (the children they had). Due to the meandering baseline, long term trends must be sought out with some effort by comparing the profile with the nearest meandering $y$ grid line. This is not the kind of variation that the visualization focuses on. We consider short and medium term fluctuations to be of greater interest in this visualization. The pattern of single-year matched fluctuations in cohort and offspring size is retained, as is the matched sinusoidal pattern of medium term booms and busts in the \nth{20} century.

\paragraph{Stacked distributions} Shaded areas depict two kinds of birth distributions. On the top axis, each polygon is the offspring of a mother cohort, indexed to the calendar years in which the births occurred. Each distribution runs from left to right, stacked atop the previous cohort distribution. Therefore, births of young mothers are those closest to the outer profile, and births of older mothers approach the baseline. On the bottom axis, each polygon is the distribution of births in an occurrence year, back-indexed to the mother cohort $x$ position on the calendar. Each distribution runs from older mothers on the left (outer profile) to younger mothers on the right (approaching baseline).

\paragraph{Shading color}
Each polygon stacked in this visualization represents a birth distribution. We would like to highlight some characteristic of these distributions that may enhance our ability to detect patterns in this series. We opt to color based on the spread of each distribution, as defined by the birth-weighted standard deviation of maternal age at birth. The top and bottom shade colors are drawn from the same colorblind-friendly palette \citep{viridis}, where light colors mark wider distributions and dark colors mark relatively compact distributions. Underlying values range from 4.9 to 6.7 for period distributions (bottom) and from 5.1 to 6.7 for cohort distributions (top). There is a clear shift from wider to narrower distributions for cohorts born after around 1910 and occurrence years after about 1960.

\paragraph{Lineage}
To aid the viewer with interpretation, we overlay a five-generation maternal lineage, which includes Alva Myrdal (born Reimer), who as much as anyone ought to remind us of the endogenous forces in the birth series we depict.\footnote{Alva Myrdal designed policies to make childbearing more compatible with women's work, to improve child well-being, and she was instrumental in other aspects of the Swedish welfare state. She also received a Nobel Peace Prize in 1982 for her work with the United Nations on disarmament, and for her influential writings on the topic of disarmament.} Since the top and bottom portions of the graph are alternative depictions of the same data, each member of this lineage appears twice: once below her mother at the same $x$ position, and once again on the top axis $x$-indexed to the year of birth and $y$ indexed to mother cohort. 

Anna Lisa was born in 1829, destined to become Alva's great grandmother. Anna Lisa gave birth to Anna Sofia in 1856. Anna Sofia therefore belongs to the offspring of the 1829 cohort, appearing directly below Anna Lisa and inside the polygon for births occurring in years 1856. The arc connecting Anna Sofia in 1829 on the bottom axis (offspring) with herself in 1856 on the top axis (appearance in birth series) is a self-link, where displacement in $x$ is equal to her mother Anna Lisa's age at birth (27), and top $y$ position places her inside the polygon for births from mothers born 1829. Anna Sofia gave birth to Alva's mother Lovisa in 1877, who gave birth to Alva in 1902, who in turn gave birth to Sissela in 1934. This descendancy continues, but births beyond Sissela are not depicted. The lineage narrative may aid the viewer in understanding how the top and bottom portions of the graph relate as two perspectives on one and the same series. 

\paragraph{Event timeline}
Several of the larger first differences in the period birth series (which cascade into cohort offspring size) have likely explanations, in most cases owing to mortality and health shocks. A selection of these are labelled, and \cite{utterstrom1954some} provides complementary discussion. These labels are included primarily to satisfy inevitable curiosity, but the feature of the data that we wish to draw attention to is the very cascading of such deviations into the series of offspring size. For example, the depression in births in 1919 due to pregnancy loss during the Spanish influenza pandemic and surge in 1920 due to recovery from the same \citep{BobergFazlic2017} is naturally an event of scientific interest. This prominent deviation is for us a discussion point for its twofold echo--- both in terms of offspring size (bottom axis) and imaginably as an amplifier of the well-known baby boom. Mothers from the 1920 cohort were the largest single contributor to cohorts born in the eight years from 1943 to 1950, the so-called first wave of the Swedish baby boom.\footnote{By rough arithmetic, we estimate that the excess size of the 1920 cohort accounts for around 1\% of first-wave (1940-1950) baby boomers in Sweden, or around 4-5\% of the excess births making the first wave of the baby boom stand out in the first place. That is, there would have been a boom anyway, but we reason it was amplified by the 1920 birth anomaly.} Only two other cohorts in this series, 1792 and 1811, might have been so dominant.\footnote{In our adjusted series, the 1792 mother cohort is the largest contributor to the nine cohorts born 1817 to 1825. The 1811 mother cohort is the largest contributor to the eight cohorts born 1837 until 1844. The 1849 cohort is the largest contributor to the six cohorts born 1877 to 1882. However, these three observations are uncertain with these data, and there is a risk it was induced by our own data adjustment described in App.~\ref{sec:cohadj}.}

\begin{figure}
\centering
[fold-out figure 4$\times$a4 paper size at 100\% in separate pdf, about here,\\ but possibly in an appendix for production reasons. The .pdf may take time to render.]
\includegraphics[scale=.3]{Figures/a4demonstration.png}
\caption{A period and cohort representation of the Swedish birth series. The top y axis indexes the births occurred in each year, broken down by mother cohort. The bottom y axis indexes the births that each mother cohort had over the course of their lives, broken down by year of occurrence. The x axis meanders proportional to a smoothed time series of the crude cohort replacement ratio. Fill color darkness is proportional to the standard deviation of the time spread of each birth count distribution: darker colors indicate more concentrated birth distributions and light colors indicate wider distributions. The birth series now appears as a flow, but reveals echoes in mother cohort size and respective total offspring size, a strong periodic series of booms and busts in recent decades, and a long term dampening of the crude replacement ratio. Five generations of a female lineage are annotated atop to serve as a guide.
}
\label{fig:foldout}
\end{figure}