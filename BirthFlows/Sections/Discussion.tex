
Several macro features of the Swedish birth series come to the fore in our visualization. These are either known features of the Swedish birth series, illustrations of some aspect of demographic thinking, or else merit further study. We briefly discuss how this visualization may inspire reflection on a set of stylized themes, including  feedback, mixture, female dominance, and reproductivity. These are for the sake of provocation, and other themes may also come to the fore in the eye of the viewer. That these themes may be evoked does not mean that our visualization would be an appropriate analytic graphic to represent the given phenomena or concept. The value of this visualization may be in the simultaneous appeal to such thinking.

\paragraph{Vibration, echoes, and cyclicity}
This visualization highlights the matched deviations in the size of cohorts and their offspring, which make the series appear to vibrate, especially in reference years before 1860. We have not found literature on the mirrored pattern of single-year deviations in cohort and offspring size. Since these are in the first place an artifact of structure, they may not be mysterious enough to study, but by extension we also know nothing of their consequences at any dimension or scale of observation. There is a literature that seeks to understand the medium term periodic fluctuations in birth cohort size \citep[This literature largely derives from][]{lee1974formal}, and more often the interplay between demographic cycles and society \citep[e.g.,][]{easterlin1987birth}. Cycles of this kind are visible in the medium-term smooth waves whose peaks (troughs) have been roughly equally spaced since 1920 (1935). The fertility and population projections we use to complete the fertility of cohorts through 2016 reveal no continuation of this pattern, but the most recent wave is all but guaranteed to continue in the form of an offspring echo. We estimate this echo will be centered on the 1990 cohort, which will end up having more offspring than any before it in Sweden. 

\paragraph{Temporal mixture}
Following the color pattern from left to right with blurred eyes, one gets a sense that the speed and rhythm of generational mixing has changed over time in Sweden, and this we think is a primary message of this visualization. To see how, consider a reference cohort as a kind of time transfer, relating a distribution of mothers to a distribution of offspring (see. Fig.~\ref{fig:juxt}). These two distributions may never overlap in human populations, due to the age of menarche\footnote{This very time lag between distributions enables simple models of human renewal to produce period cycles \citep{wachter1991pre}.}, and their respective compactness may vary independently. At least two macro patterns in the distributions can be seen through the color pattern: i) The series undergoes a structural shift from light hues coloring wider and more overlapped distributions to dark hues coloring narrower and more steeply stacked distributions. This shift is centered around reference year 1940 --- that is, on the mother cohort 1910 or occurrence year 1965. ii) Gentle undulations approximately 30 years (or a generation) apart prevail through both the early and late parts of the series, but their amplitude is dwarfed by the previous mentioned structural shift. 

\paragraph{A female dominant view}
These observations and conjectures are predicated on the assumption that father age distributions either do not matter or else that they change in sync with mother age distributions. This begs the question as to whether the revealed patterns (vibration, replacement, mixture) are preserved, dampened, or magnified when indexed instead to father cohorts. Such sex comparisons have been made for migration-adjusted male and female net reproductivity \citep{hyrenius1951reproduction} in Sweden, with males' net reproductivity often higher than females' between 1850 and 1950. However, we do not know how the degree of first difference reflection (vibration) would compare for males and females: if the correlation turned out to be stronger for males, what would that say about the common assumption of female dominance in models of human renewal? Certainly the joint consideration of both mothers' and fathers' cohorts would lead to a more complete temporal mixture than what we may surmise from mothers' cohorts alone: Changes in age heterogamy over time would lead to a change in the pattern of temporal mixture, with more age-homogamous parentage leading to more directed time transfers in much the same way as compact birth distributions. Such questions could be investigated to some degree using large genealogical databases.

\paragraph{Reproductivity}
Birth cohort sizes more than doubled from the mid 18th Century until the late 19th Century in Sweden, after which time periodic fluctuations in size have ranged between 80000 and 140000, with no obvious long term trend of growth or decrease in the last century or over our projected horizon. Offspring size doubled from the mid 18th Century until the mid 19th Century, falling until around 1900, after which time it has grown in periodic spurts, with the 1990 cohort's offspring projected to be the largest in our series at over 150000. These two series relate in our baseline pattern of crude generation replacement, which shows several clear turning points. In the graph, the baseline relates to the transecting horizontal line, with points above showing crude growth and points below crude decrease. This measure of replacement reached a maximum in reference year 1822, a series minimum in 1901, with a projected local maximum around 1999. We reckon it impossible for a trained demographer to gaze at this visualization without either embracing or fighting against it as a literal representation of Sweden's reproductivity. As a picture of births alone, there is no account of population stocks in general, of attrition from mortality or emigration, or of increment through immigration. Accounting for any of these other phenomena would involve transformations away from the scale of absolute births and towards something closer to demographic behavior. There are other measures of reproductivity that take into account mortality \citep{kuczynski1932fertility}, or both mortality and migration \citep[][inter alia]{hyrenius1951reproduction,ortega2007birth,preston2007intrinsic,wilson2013migration,ediev2014new}, and which could therefore offer alternative values to arrange on the Lexis grid to recreate this work. Structure-driven shocks would in this case likely diminish or disappear, whereas behavior driven shocks might amplify, but the corresponding visualization would be one of forces or behavior rather than our tangible unit of births. 

\paragraph{Questions}
These reflections can be translated into a set of tangible questions of varying scope and import. For example: What are the consequences of short term anomalies in cohort and offspring size, such as the 1920 cohort or those prior to the 20th Century? How do deviations in offspring size scale with the size of cohort size fluctuations? How do period and cohort cycles in the \emph{spread} of the birth distribution relate, and what causes and consequences do they have? How would this picture (and its derived questions) change if indexed instead to father cohorts? What if the value depicted were rates rather than counts? What if the value depicted were stocks at some other age, but indexed in the same way? We think that it may provoke many more questions than these.
