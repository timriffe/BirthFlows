One of the most immediately visible features of Fig.~\ref{fig:foldout} is the propagation of first differences in $B(t)$ to $B(c)$. The 1920 cohort is a particularly visible example: There were 23560 more births in 1920 than in 1919, an increase of 20.4\%, and mothers from the 1920 cohort also gave birth to 20.7\% more babies than the 1919 cohort. Fig.~\ref{fig:rBcrBt} displays the relationship in proportional first differences between matched birth cohort and offspring size. For the most part, the size of such structural echoes is maintained 1:1 in cohort offspring.


\begin{figure}
% fig from StructuralEchoes.R
\includegraphics[scale=.6]{Figures/rBcrBt.pdf}
\caption{A roughly 1:1 dose-response relationship in relative size of structural echo.}
\label{fig:rBcrBt}
\end{figure}