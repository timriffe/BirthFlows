%%This is a very basic article template.
%%There is just one section and two subsections.
\documentclass{article}
\usepackage[a4paper, margin=2.5cm]{geometry}
\usepackage{amsmath}
\usepackage{caption}
\usepackage{placeins}
\usepackage{graphicx}
\usepackage{subcaption}
\usepackage{setspace}
\usepackage{float}
\usepackage{wrapfig}
\usepackage{pdfpages}
%\usepackage[active,tightpage]{preview}
\usepackage{natbib}
\bibpunct{(}{)}{,}{a}{}{;} 
\usepackage{url}
\usepackage{nth}
\usepackage{authblk}
\usepackage{blindtext}
% for the d in integrals
\newcommand{\dd}{\; \mathrm{d}}
\newcommand{\tc}{\quad\quad\text{,}}
\newcommand{\tp}{\quad\quad\text{.}}
\newcommand{\ra}{\rightarrow}
\def\lsub#1#2%
  {\mathop{}%
   \mathopen{\vphantom{#2}}_{#1}%
   \kern-\scriptspace%
   #2}
\def\lsup#1#2%
  {\mathop{}%
   \mathopen{\vphantom{#2}}^{#1}%
   \kern-\scriptspace%
   #2}

\defcitealias{HMD}{HMD}

\newcommand\ackn[1]{%
  \begingroup
  \renewcommand\thefootnote{}\footnote{#1}%
  \addtocounter{footnote}{-1}%
  \endgroup
}
\begin{document}

%\title{Macro patterns in the shape of aging}
\title{Boom, echo, pulse, flow}
\author[1]{Tim Riffe\thanks{riffe@demogr.mpg.de}}
\author[1]{Kieron Barclay}
\affil[1]{Max Planck Institute for Demographic Research}
\maketitle

\begin{abstract}
Human population renewal starts with births. Since births can happen at any
time in the year and over a wide range of ages, demographers typically imagine
the birth series as a continuous flow. Taking this construct literally, we
visualize the birth series as a flow. A long birth series allows us to
juxtapose the children born in a particular year with the children that
they in turn had over the course of their lives, yielding a crude notion of
cohort replacement. Macro patterns in generational growth define the meandering
path of the flow, while temporal booms and busts echo through the flow with the
regularity of a pulse.
\end{abstract}

\onehalfspacing
\section{Introduction}
Usually we think of fertiltiy as an age-regulated process. In any case it is
bounded by menarche and menopause, both of which are anchored to age. These anchors may
move, but not far or fast. And between these bounds fertility patterns appear to
conform to some regular schema, best captured by fertility rates. In this treatment, we retreat
from rates, the material of projections, to babies, the raw material of
population renewal. A picture of the births in a year is for demographers most
instinctively broken down by the age of mothers who gave birth in that year,
Fig.~\ref{fig:agemother}, or by the year of birth of mothers Fig.~\ref{fig:cohmother}. These two distributions are essentially identical, but appear as mirror images if chronological time is enforced in $x$.

\begin{figure}[ht!]
\begin{subfigure}[t]{0.5\textwidth}
        \centering
        \includegraphics[width=\textwidth]{Figures/Fig11900MotherAge.pdf}
        \caption{Births in 1900 by age of mother}
        \label{fig:agemother}
\end{subfigure}
~
\begin{subfigure}[t]{0.5\textwidth}
        \centering
        \includegraphics[width=\textwidth]{Figures/Fig11900MotherCohort.pdf}
        \caption{Births in 1900 by year of birth of mother}
          \label{fig:cohmother}
\end{subfigure}
\caption{Births in a year structured by mothers' age versus mothers' year of birth are a
reflection over $y$ and shift over $x$.}
\end{figure}

If one disposes of a long-enough time series of births classified by mothers' year of birth, then one may further examine and break down the full reproductive career of the cohort of individuals born in a particular year. As the childbearing of a cohort is spread over a synchronous span of ages and years. Since age and calendar time are synchronous for a cohort, the classification by age (Fig.~\ref{fig:age1900mother}) or year (Fig.~\ref{fig:year1900}) yields identical and redundant distributions.

\begin{figure}[ht!]
\begin{subfigure}[t]{0.5\textwidth}
        \centering
        \includegraphics[width=\textwidth]{Figures/Fig11900IDAge.pdf}
        \caption{Births from mothers born in 1900 by age of mother}
        \label{fig:age1900mother}
\end{subfigure}
~
\begin{subfigure}[t]{0.5\textwidth}
        \centering
        \includegraphics[width=\textwidth]{Figures/Fig11900IDYear.pdf}
        \caption{Births from mothers born in 1900 by year}
          \label{fig:year1900}
\end{subfigure}
\caption{Births of a cohort structured by mothers' age versus mothers' year of birth are a
reflection over $y$ and shift over $x$.}
\end{figure}

The births in a year are classified by mothers' cohort, i.e. cohort \emph{origins} in Fig.~\ref{fig:cohmother}, whereas the births \emph{from} a cohort are classified \emph{to} time in Fig.~\ref{fig:year1900}. The two distributions are different in kind, but relatable and both on a common scale. A fuller representation of their relationship would place them as two disjoint distributions on the same timeline, as in Fig~\ref{fig:juxt}.

\begin{figure}[ht!]
 \centering
        \includegraphics[width=\textwidth]{Figures/Fig31900juxt.pdf}
        \caption{Births from mothers born in 1900 by year}
          \label{fig:juxt}
\end{figure}

The two distributions in Fig.~\ref{fig:juxt} are related, and of comparable scale, but different in kind. The $x$ coordinate of the left distribution is indexed to mothers's birth cohort, whereas the $x$ coordinate of the right distribution is indexed to child cohort, ocurrence year. In this way the $x$ coordinates belong to grandmothers and grandchildren, where the \emph{ego} generation is 1900. These are two quantities that we may wish to compare in various ways to get a better feel and understanding of the Swedish birth series. 

For the case of these Swedish data, we have 241 such distribution pairs, making single-axis rendering impractical. An honest attempt might look like Fig.~\ref{fig:reflect1}, where we reflect the Fig.~\ref{fig:juxt} left distribution over $y$ (\textbf{A}), keeping the Fig.~\ref{fig:juxt} right-side distribution on top (\textbf{B}). These two distributions are linked by the year 1900, which of course overlaps with neither of them. In this representation, \textbf{A} and \textbf{B} are re-drawn for each possible ego year (1775-2015), and therefore imply a large sequential set of overlapping distributions. Each \nth{20} distribution is highlighted, that is, this is an honest attempt to make this graph legible, but i) the high degree of overlapping and ii) the spatial dissociation of each \textbf{A} --- \textbf{B} pair makes the intended comparison difficult over the series.

\begin{figure}[ht!]
 \centering
        \includegraphics[width=\textwidth]{Figures/FxFlowReflect.pdf}
        \caption{Two time series of birth count distributions. The top series is composed of offspring distributions, indexed to ocurrence years. The bottom series is composed of birth cohorts indexed by mothers' birth cohorts. \textbf{B} is the offspring of mothers from the 1900 cohort indexed in $x$ to ocurrence year, and \textbf{A} are the births ocurred in 1900 indexed in $x$ to mothers' birth cohorts. The cross-section \textbf{a} gives \textbf{A} and the cross-section \textbf{b} gives \textbf{B}.}
          \label{fig:reflect1}
\end{figure}

Fig.~\ref{fig:reflect1} produces at least two noteworthy artifacts that we may wish to preserve or clarify. 1) First order differences in the top series appear to cascade into the lower series--- This merely points out that larger cohorts have more offspring than smaller neighboring cohorts and vice versa, sudden fertility rate changes notwithstanding. 2) The composition of \textbf{A} in the bottom series is implied by the cross-section \textbf{a} of the top series, and the composition of  \textbf{B} is implied by the cross-section \textbf{b}. To reiterate: \textbf{A} are all the births in 1900 indexed back to mothers' cohorts. Each possible \emph{slice} of \textbf{A} comes from a different top distribution as it crosses the year 1900 (and vice versa for the bottom). \textbf{a} and \textbf{b} are in a sense already juxtaposed for us, as they share an $x$ coordinate. The ``problem'' with the cross-sections \textbf{a} (\textbf{b}) is that each \emph{slice} of the corrsponding distribution \textbf{A} (\textbf{B}) is perfectly overlapped, such that it is just about impossible to imagine what \textbf{A} might look like if presented only with \textbf{a} and its surroundings. 
% Q: should the 'pointers' for a and b actually be drawn on top of the data series, as of a cross-section?
\pagebreak
% text above pushes whole thing down
\begin{wrapfigure}{r}{0.5\textwidth}
 \centering
        \includegraphics[width=3in]{Figures/FigReflection.pdf}
        \caption{The 1900 cohort as a composite bar with its offspring reflected over $y$. The size of each bar stacked in the top composition is proportionate to the area of its corresponding polygon in the left distribution of Fig.~\ref{fig:juxt}. The size of each bar stacked in the lower composition is proportionate to the area of its corresponding polygon in the right distribution of Fig.~\ref{fig:juxt}.}
          \label{fig:refl}
\end{wrapfigure}
% text under goes to the left then builds down
In this way the two distributions that we might wish to compare for a given ego year are already available at a like coordinate, but comparison is stifled by overplotting. If instead we stack the slices that are indecipherably overlapped in \textbf{a} (and likewise for \textbf{b}) we get something like that shown in Fig.~\ref{fig:refl}, cumulative birth distributions.\footnote{Young mothers are on top and older mothers on bottom for both distributions. It would also make sense to plot increasing (or decreasing) ages eminating out from the centerline in both directions.} Here the total bar length is proportional to the total cohort (offspring) size, and stacked bins reflect 5-year mother cohorts (ocurrence years). From this representation it is clear that mothers born in the 20 years between 1860 and 1880 produced the bulk of the 1900 cohort (86\%), which itself produced the majority of its offspring in the 20 years between 1920 and 1940 (90\%). It is also quite visible that the 1900 cohort did not replace itself in a crude sense: 138,139 babies formed a cohort whose mothers gave birth to 95,379 babies over their lifecourse, a crude replacement of 69\%. Other perspectives on reproduction that account for survival would give a more optimistic assessment. The key feature of Fig.~\ref{fig:refl} is that the two distributions that were disjoint in Fig.~\ref{fig:juxt} and hard to pick out in Fig.~\ref{fig:reflect1} can now be associated at a common $x$ coordinate. This virtue allows us to view the time series of Fig.~\ref{fig:reflect1} with greater clarity and perhaps reveal some macro properties of the history of Swedish natality.

Fig.~\ref{fig:foldout} is a depiction of the exercise of Fig.~\ref{fig:refl}, cohort bars on top reflected with offspring bars on the bottom. Equal bounded bins from Fig.~\ref{fig:refl} are joined into continuous regions. For the top region, filled polygons represent the births of mothers from quinquennial cohorts, spread over time. For the bottom region, filled polygons represent the mother-cohort origins of the births in quinquennial periods. The darkness and saturation of polygon fill colors are approximately proportional to the total births in the polygon (and therefore true to grayscale printing), whereas hue is irrelevant. In this way, darkness and saturation on the top are proportional to total height (with respect to baseline) on the bottom, and vice versa. 

The meandering baseline of Fig.~\ref{fig:foldout} is proportional to a smoothed time series of the crude cohort replacement rate. We overlay a horizontal line to indicate periods of approximate growth, replacement, and contraction. Periods where the meandering $x$-baseline is above this line (ca 1780 to 1860) indicate crude growth, and periods below the horizontal reference line (ca 1870 to 1930) indicate crude generation contraction. 

To aid the viewer with interpretation, we overlay a known lineage of five female generations,\footnote{This lineage can be located in the public domain on \url{https://www.geni.com/people/Karin-Ottolina-Landsten/6000000022470480183}.} where $x$ position is exact to the year, $y$ position in the top region is matched to the mothers cohort, and $y$ position in the bottom is matched to daughters' year of birth. Wider horizontal spacing between generations over time indicates increasing ages at maternity within this lineage (increasing from 23 to 39).

Several macro features come to the fore in this visualization. These are either known features of the Swedish birth series, or else mention further study. Echoes, booms, why is boom periodicity a recent phenomenon? Is there a dose-response to vertical reverberation in first derivative (can this be referred to as first or second order?) features.

To be continued.

\begin{figure}
\centering
[fold-out figure 125cm wide by 30cm tall at 100\% in separate pdf, about here.]
%\includegraphics[scale=.9]{Figures/SwedenBirthFlowsManFoldout.pdf}
%\includepdf[landscape]{Figures/SwedenBirthFlowsManFoldout.pdf}
\caption{A time series of the same graphical construct as presented in Fig.~\ref{fig:refl}. The $x$ axis now meanders proportional to a smoothed time series of the crude cohort replacement rate. Fill color darkness and saturation are approximately proportional to the total number of births in each birth distribution. The birth series now appears as a flow, but reveals echoes in cohort and offspring size, an odd periodicity in recent decades, and a long term dampening of the crude replacement rate. A 5-generation female lineage is annotated atop to serve as a guide.}
\label{fig:foldout}
\end{figure}


\singlespacing
\bibliographystyle{plainnat}
  \bibliography{references} 
\end{document}
