%%This is a very basic article template.
%%There is just one section and two subsections.
\documentclass{article}
\usepackage[a4paper, margin=2.5cm]{geometry}
\usepackage{amsmath}
\usepackage{caption}
\usepackage{placeins}
\usepackage{graphicx}
\usepackage{subcaption}
\usepackage{setspace}
\usepackage{float}

%\usepackage[active,tightpage]{preview}
\usepackage{natbib}
\bibpunct{(}{)}{,}{a}{}{;} 
\usepackage{url}
\usepackage{nth}
\usepackage{authblk}
% for the d in integrals
\newcommand{\dd}{\; \mathrm{d}}
\newcommand{\tc}{\quad\quad\text{,}}
\newcommand{\tp}{\quad\quad\text{.}}
\newcommand{\ra}{\rightarrow}
\def\lsub#1#2%
  {\mathop{}%
   \mathopen{\vphantom{#2}}_{#1}%
   \kern-\scriptspace%
   #2}
\def\lsup#1#2%
  {\mathop{}%
   \mathopen{\vphantom{#2}}^{#1}%
   \kern-\scriptspace%
   #2}


\defcitealias{HMD}{HMD}

\newcommand\ackn[1]{%
  \begingroup
  \renewcommand\thefootnote{}\footnote{#1}%
  \addtocounter{footnote}{-1}%
  \endgroup
}
\begin{document}

%\title{Macro patterns in the shape of aging}
\title{Boom, echo, pulse, flow}
\author[1]{Tim Riffe\thanks{riffe@demogr.mpg.de}}
\affil[1]{Max Planck Institute for Demographic Research}
\maketitle

\begin{abstract}
Human population renewal starts with births. Since births can happen at any
time in the year and over a wide range of ages, demographers typically imagine
the birth series as a continuous flow. Taking this construct literally, we
visualize the birth series as a flow. A long birth series allows us to
juxtapose the children born in a particular year with the children that
they in turn had over the course of their lives, yielding a crude notion of
cohort replacement. Macro patterns in generational growth define the meandering
path of the flow, while temporal booms and busts echo through the flow with the
regularity of a pulse.
\end{abstract}

\section{Introduction}
Usually we think of fertiltiy as an age-regulated process. In any case it is
bounded by menarche and menopause, both of which are anchored to age. These may
move, but not far or fast. And between these bounds fertility patterns appear to
conform to some regular schema, best captured by fertility rates. We retreat
from rates, the material of projections, to babies, the raw material of
population renewal. A picture of the births in a year is for demographers most
instinctively broken down by the age of mothers who gave birth in that year,
Fig.~\ref{fig:agemother}, or by the year of birth of mothers Fig.~\ref{fig:cohmother}. These two distributions are essentially identical, but appear as mirror images if chronological time is enforced in $x$.

\begin{figure}[ht!]
\begin{subfigure}[t]{0.5\textwidth}
        \centering
        \includegraphics[width=\textwidth]{Figures/Fig11900MotherAge.pdf}
        \caption{Births in 1900 by age of mother}
        \label{fig:agemother}
\end{subfigure}
~
\begin{subfigure}[t]{0.5\textwidth}
        \centering
        \includegraphics[width=\textwidth]{Figures/Fig11900MotherCohort.pdf}
        \caption{Births in 1900 by year of birth of mother}
          \label{fig:cohmother}
\end{subfigure}
\caption{Births in a year structured by mothers' age versus mothers' year of birth are a
reflection over $y$ and shift over $x$.}
\end{figure}

If one disposes of a long-enough time series of births classified by mothers' year of birth, then one may further examine and break down the full reproductive career of the cohort of individuals born in a particular year. As the childbearing of a cohort is spread over a synchronous span of ages and years. Since age and calendar time are synchronous for a cohort, the classification by age (Fig.~\ref{fig:age1900mother}) or year (Fig.~\ref{fig:year1900}) yields identical and redundant distributions.

\begin{figure}[ht!]
\begin{subfigure}[t]{0.5\textwidth}
        \centering
        \includegraphics[width=\textwidth]{Figures/Fig11900IDAge.pdf}
        \caption{Births from mothers born in 1900 by age of mother}
        \label{fig:age1900mother}
\end{subfigure}
~
\begin{subfigure}[t]{0.5\textwidth}
        \centering
        \includegraphics[width=\textwidth]{Figures/Fig11900IDYear.pdf}
        \caption{Births from mothers born in 1900 by year}
          \label{fig:year1900}
\end{subfigure}
\caption{Births of a cohort structured by mothers' age versus mothers' year of birth are a
reflection over $y$ and shift over $x$.}
\end{figure}

The births in a year are classified by mothers' cohort, i.e. cohort \emph{origins} in Fig.~\ref{fig:cohmother}, whereas the births \emph{from} a cohort are classified \emph{to} time in Fig.~\ref{fig:year1900}. The two distributions are different in kind, but relatable and both on a common scale. A fuller representation of their relationship would place them as two disjoint distributions on the same timeline, as in Fig~\ref{}.

\begin{figure}[ht!]
 \centering
        \includegraphics[width=\textwidth]{Figures/Fig31900juxt.pdf}
        \caption{Births from mothers born in 1900 by year}
          \label{fig:juxt}
\end{figure}

\FloatBarrier
\singlespacing
\bibliographystyle{plainnat}
  \bibliography{references} 
\end{document}
