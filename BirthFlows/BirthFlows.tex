%%This is a very basic article template.
%%There is just one section and two subsections.
\documentclass{article}
\usepackage[a4paper, margin=2.5cm]{geometry}
\usepackage{amsmath}
\usepackage{caption}
\usepackage{placeins}
\usepackage{graphicx}
\usepackage{subcaption}
\usepackage{setspace}
\usepackage{float}

%\usepackage[active,tightpage]{preview}
\usepackage{natbib}
\bibpunct{(}{)}{,}{a}{}{;} 
\usepackage{url}
\usepackage{nth}
\usepackage{authblk}
% for the d in integrals
\newcommand{\dd}{\; \mathrm{d}}
\newcommand{\tc}{\quad\quad\text{,}}
\newcommand{\tp}{\quad\quad\text{.}}
\newcommand{\ra}{\rightarrow}
\def\lsub#1#2%
  {\mathop{}%
   \mathopen{\vphantom{#2}}_{#1}%
   \kern-\scriptspace%
   #2}
\def\lsup#1#2%
  {\mathop{}%
   \mathopen{\vphantom{#2}}^{#1}%
   \kern-\scriptspace%
   #2}


\defcitealias{HMD}{HMD}

\newcommand\ackn[1]{%
  \begingroup
  \renewcommand\thefootnote{}\footnote{#1}%
  \addtocounter{footnote}{-1}%
  \endgroup
}
\begin{document}

%\title{Macro patterns in the shape of aging}
\title{Boom, echo, pulse, flow}
\author[1]{Tim Riffe\thanks{riffe@demogr.mpg.de}}
\affil[1]{Max Planck Institute for Demographic Research}
\maketitle

\begin{abstract}
Human population renewal starts with births. Since births can happen at any
time in the year and over a wide range of ages, demographers typically imagine
the birth series as a continuous flow. Taking this construct literally, we
visualize the birth series as a flow. A long birth series allows us to
juxtapose the children born in a particular year with the children that
they in turn had over the course of their lives, yielding a crude notion of
cohort replacement. Macro patterns in generational growth define the meandering
path of the flow, while temporal booms and busts echo through the flow with the
regularity of a pulse.
\end{abstract}

\section{Introduction}
It has long been pointed out that human reporoduction is a highly age-regulated
process \citep[e.g.,][]{kuczynski1932fertility}

\FloatBarrier
\singlespacing
\bibliographystyle{plainnat}
  \bibliography{references} 
\end{document}
