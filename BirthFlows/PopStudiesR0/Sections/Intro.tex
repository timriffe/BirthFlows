Usually demographers think of fertility as an age-regulated process. In any case it is bounded by menarche and menopause, both of which are anchored to biological age. These anchors may move, but not far or fast. And between these bounds, at least within acceptably homogeneous subpopulations, fertility patterns appear to conform to some regular schema. Since births can happen at any time throughout the year and throughout the fertile age range, and since demography usually deals in large numbers, it is common to imagine births at the population level as a continuous stream or flow. This is so not only as a pragmatic assumption to allow for calculus, but it also gives us a heuristic understanding of fertility as a smoother of population structure \citep{arthur1982ergodic}. In the present exposition, we retreat from rates, the basis of projections and stable population theory, to the absolute number of babies born, the raw material of population renewal. 

We aim to represent a long time series of birth counts as a multilayered view of population renewal, a perspective enabled by Sweden's long history of population statistics. The birth series is rendered as a flow, in such a way as to simultaneously suggest several analytic perspectives, and to invite newcomers and curious minds deeper into the discipline of demography. This image, a large fold-out insert, entails investment from the viewer, and this manuscript serves as a protracted legend and caption. It is not a concise analytic plot, but rather a composite of hundreds of distributions, rendered both in the period and cohort perspectives. Intellectual payoffs include a simultaneous sense of long term patterns of generational mixing and generational replacement, medium term baby booms and echos, and the short term shocks of population momentum. %We challenge more experienced demographers to relate this image to the Lexis diagram, to imagine how the picture would change if fertility were indexed to fathers' age, or to reimagine this image of aggregates as an immense set of lineages.

Our time series of birth counts stretches from 1735 until 2016, and it is augmented by a projection of the completed fertility of cohorts through 2016, bringing the latest year of birth to 2071. The temporal spread from the earliest mother cohort in our final data set (1687) to the latest year of birth occurrence (2071) is 385 years. We describe our input data and adjustments to it briefly in Sec.~\ref{sec:data}, and in a detailed set of appendices. Sec.~\ref{sec:birthdist} relates different age-structured birth count distributions to a common calendar. These distributions become the basic elements of our visualization. Sec.~\ref{sec:description} gives a plain language description of how to interpret the visualization and a guide through some of its major features. Sec.~\ref{sec:disc} discusses the strengths and limitations of this particular visualization, and Sec.~\ref{sec:conc} concludes with a summary of this work.

